\documentclass[../paper.tex]{subfiles}
\usepackage{drftcite}

% Document
\begin{document}
    Before we start talking about the different methods we can use to predict the weather, we first need to know where we can get the data from or where the data is coming from that we use to train our models.
    \\\\
    There are a lot of different sources for weather data, some of them are free, some of them are paid, and because the methods we will look at in this paper make use of similar data, we will talk a bit about it here.
    \\\\
    Some of the data sources we will look at are:
    \begin{itemize}
        \item ECMWF
        \item IoT
        \item etc.
    \end{itemize}
    There are a lot more sources that some of the methods we will look at use, but we will only look at the ones that are widely used.
    \subsubsection{ECMWF}
    The European Centre for Medium-Range Weather Forecasts (ECMWF) is an independent intergovernmental organisation supported by most of the nations of Europe and is based at Shinfield Park, Reading, United Kingdom.
    The center's operational forecasts are produced from its ``Forecast System'' (sometimes informally known in the United States as the ``European model'') which is run every twelve hours and forecasts out to ten days\cite{b4}.
    ECMWF is also developing their own digital twin, to predict weather and climate change\cite{b5}.
    \\\\
    ECMWF makes wide range os data available to the public.
    The data that are becoming available are based on a range of high-resolution forecasts (HRES \- 9 km horizontal resolution) and ensemble forecasts (ENS \- 18 km horizontal resolution).
    They will be made accessible at a resolution of 0.4 x 0.4 degrees \cite{ecmwf_data_free}.
    \\\\
    An overview of what kinds of data are available is given on the ECMWF website.
    More detailed explanations of how to access the data are also available\cite{ecmwf_data_free}.
    \subsubsection{IoT}
    The Internet of Things (IoT) plays a significant role in collecting data for weather forecasting services.
    Here's how it works:
    \begin{itemize}
        \item \textbf{Sensors}: IoT uses a network of sensors to collect real-time data on various weather parameters such as temperature, pressure, humidity, wind speed, and light intensity\cite{iot1, iot4}.
        These sensors can be embedded in various devices and locations, including weather stations, agricultural fields, and even moving vehicles\cite{iot1, iot4}.
        \item \textbf{Data Transmission}: The data collected by these sensors is then transmitted over the internet to a central monitoring system or cloud-based servers\cite{iot1, iot3}.
        This allows for real-time monitoring and accessibility of data from anywhere in the world\cite{iot1}.
        \item \textbf{Data Analysis}: The collected data is analyzed and processed using advanced algorithms and machine learning techniques.
        This helps in predicting weather patterns and changes more accurately\cite{iot1, iot3}.
        \item \textbf{Alerts and Notifications}: IoT systems can also be programmed to send alerts and notifications based on the data collected.
        For instance, they can warn about sudden and drastic weather changes\cite{iot1}.
        \item \textbf{Integration with Weather Models}: The data collected through IoT is often integrated with other weather forecasting models to improve their accuracy and reliability\cite{iot3}.
    \end{itemize}

    In summary, IoT brings a new era of weather forecasting by enabling the collection of real-time, hyper-local data, which significantly enhances the accuracy of weather predictions\cite{iot4}.

\end{document}