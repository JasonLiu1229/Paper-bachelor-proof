\documentclass[../paper.tex]{subfiles}

% Document
\begin{document}
    We have looked at a variety of different methods for simulating weather and climate.
    We have looked at STEP, OpenWeatherMap, Meteomatics, ClimaX, and GraphCast.
    \\\\
    There are many available methods for simulating weather and climate. \\
    Some of them are more accurate than others, some of them are faster than others, and some of them are easier to use than others. \\
    This includes trained models, APIs, and algorithms. \\
    \\\\
    STEP is an algorithm for computing ensemble forecasts.
    It is the standard method used in the industry for computing ensemble forecasts by a good deal of different weather forecasting agencies, including the European Centre for Medium-Range Weather Forecasts (ECMWF).
    \\\\
    OpenWeatherMap and Meteomatics are APIs for accessing weather data.
    They are both easy to use, and provide a lot of different current and historical data. \\
    Both having their pros and cons, but overall, they both have a free plan, which makes them both viable options for our application. \\
    \\\\
    ClimaX and GraphCast are both machine learning models for weather and climate forecasting.
    They are both very accurate, and very fast. \\
    They both provide pretrained models, which makes them easier to implement in our application. \\
    \\\\
    In general, these methods are all viable options for our application and knowing which one to use depends on the specific use case.
    In case we want to be fully in control of the data, we can use STEP. \\
    In case we want to use machine learning, we can use ClimaX or GraphCast, and make use of the pretrained models. \\
    In case we want to use an API, we can use OpenWeatherMap or Meteomatics.
\end{document}