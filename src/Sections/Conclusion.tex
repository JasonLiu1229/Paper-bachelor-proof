\documentclass[../paper.tex]{subfiles}

% Document
\begin{document}
    We have looked at a variety of different methods for simulating weather and climate.
    We have looked at STEP, OpenWeatherMap, Meteomatics, ClimaX, and GraphCast, but what is the best method for our application?
    \\\\
    So if we look at the algorithms, we can see that this is not a good choice for our application.
    We want an easy-to-use method, and this is not the case for algorithms, because we have to implement the algorithm ourselves and make sure that it works.
    \\\\
    Now let's look at the APIs.
    Both OpenWeatherMap and Meteomatics are easy to use, and provide a lot of different current and historical data.
    The biggest downside of these APIs is that they are very limited in their free plan, in case we want to make more than 500 calls per day (1000 for OpenWeatherMap), we have to upgrade to a paid plan.
    Not only that having access to deep historical data is not possible with these APIs, if we don't want to pay for it.
    \\\\
    Now let's look at the machine learning models.
    Both ClimaX and GraphCast are very accurate, and very fast.
    Both are pretrained and easy to implement in our application.
    The biggest downside of these models is that they are black box models, meaning that it is not possible or hard to understand how they work.
    (Not a downside that affects our application)
    \\\\
    In general, the combination of an API and a machine learning model is the best option for our application.
    We can make use of the API to get the current weather data and slowly build up our historical data.
    In case the amount of API calls is not enough, we can make use of the machine learning model to predict the weather.
    Also, in case we want concurrent weather data, we can make use of the machine learning model.
    \\\\
    A good combination would be OpenWeatherMap and ClimaX\@.
    This is because OpenWeatherMap has a free plan with 1000 calls per day (usually enough) and ClimaX is a well tested and accurate model.
    A good alternative would be Meteomatics and ClimaX, this is because Meteomatics has pre\-made connectors to different program languages, which makes it easier to implement in our application.
    Also, it has a python package that can be used to make API calls.
    So in regard to implementation, Meteomatics is a better choice than OpenWeatherMap.
\end{document}