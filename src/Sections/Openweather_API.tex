\documentclass[../paper.tex]{subfiles}

% Document
\begin{document}
    OpenWeatherMap is a weather API that provides weather data for any location on the globe.
    It uses machine learning (ML) to significantly advance both the accuracy and computing speed of global assemble forecasting models, a practice that was impossible only a few years ago\cite{b1}.
    \\\\
    OpenWeatherMap offers a variety of APIs, including the One Call API 3.0, which provides current weather and forecasts, minute forecast for 1 hour, hourly forecast for 48 hours, daily forecast for 8 days, and government weather alerts\cite{b1}.
    The API also provides weather data for any timestamp for 40+ years historical archive and 4 days ahead forecast, daily aggregation of weather data for 40+ years archive and 1.5 years ahead forecast, hourly forecast for 4 days, 16 days forecast, and climatic forecast for 30 days\cite{b1}.
    \\\\
    In addition, OpenWeatherMap provides beautiful multi-layer maps that create the visual perception of weather.
    You can choose from a set of OpenWeather Model layers such as wind, temperature, pressure, and others, or select radar data for a detailed precipitation picture\cite{b1}.

    \subsubsection{Data}
    Openweathermap makes use of different soureces of data, such as:
    \begin{itemize}
        \item \textbf{ECMWF} - The European Centre for Medium-Range Weather Forecasts (ECMWF) is an independent intergovernmental organisation supported by most of the nations of Europe and is based at Shinfield Park, Reading, United Kingdom.
         The center's operational forecasts are produced from its ``Forecast System'' (sometimes informally known in the United States as the ``European model'') which is run every twelve hours and forecasts out to ten days\cite{b4}.
         ECMWF is also developing their own digital twin, to predict weather and climate change\cite{b5}.
        \item \textbf{NOAA} - The National Oceanic and Atmospheric Administration (NOAA) is an American scientific agency within the United States Department of Commerce that focuses on the conditions of the oceans, major waterways, and the atmosphere\cite{b6}.
        \item \textbf{More} - Openweathermap also makes use of other sources of data, such as: GFS, NEMS, ICON, AROME, and others\cite{b2}.
    \end{itemize}
    Because of these data sources Openweathermap is able to provide accurate weather data for any location on the globe.

    \subsubsection{Advantages}
    Here are some of the advantages of OpenWeatherMap
    \begin{itemize}
        \item \textbf{Accuracy and speed} - OpenWeatherMap uses machine learning to significantly advance both the accuracy and computing speed of global assemble forecasting models, a practice that was impossible only a few years ago\cite{b1}.
        \item \textbf{Global coverage} - OpenWeatherMap provides weather data for any location on the globe\cite{b1}.
        \item \textbf{Historical data} - OpenWeatherMap provides weather data for any timestamp for 40+ years historical archive and 4 days ahead forecast, daily aggregation of weather data for 40+ years archive and 1.5 years ahead forecast, hourly forecast for 4 days, 16 days forecast, and climatic forecast for 30 days\cite{b1}.
        \item \textbf{Multi-layer maps} - OpenWeatherMap provides beautiful multi-layer maps that create the visual perception of weather\cite{b1}.
    \end{itemize}

    \subsubsection{Disadvantages}
    Here are some of the disadvantages of OpenWeatherMap
    \begin{itemize}
        \item \textbf{Limited free plan} - OpenWeatherMap offers a free plan that allows 60 calls per minute, one million calls per month, and 5-day forecast, but it does not include historical data\cite{b1}.
    \end{itemize}

    \subsubsection{Motivation}
    OpenWeatherMap is a weather API that provides accurate weather data for any location on the globe.
    Making use of the STEP algorithm explained previously.
    Combining this algorithm with machine learning (ML) to significantly advance both the accuracy and computing speed.
    \\\\
    Also with the free plan we get up to 60 calls per minute, one million calls per month, and 5-day forecast, which is more than enough for our application.
    Although the free plan doesn't include historical data, we can save the data and make our own historical data.
    The Downside of this approach is that we have to wait for the data to be collected.
    \\\\
    In case we want forecast data for more than 5 days, we have to implement our own model to predict the weather (making use of the collected historical data), or upgrade to a paid plan.
\end{document}