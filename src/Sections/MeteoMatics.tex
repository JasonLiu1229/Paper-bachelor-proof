\documentclass[../paper.tex]{subfiles}

% Document
\begin{document}
    Meteomatics is a weather API that provides weather data for any location on the globe.
    It works in a similar way to OpenWeatherMap, but it has some different advantages and disadvantages.

    \subsubsection{vs Openweathermap}
    Because it works in a similar way to OpenWeatherMap, we will only mention the differences between the two.
    Meteomatics offers more features in its free plan, such as historical data of the past 24 hours, but the free plan is limited to 500 calls per day.
    \\\\
    Another advantage is that Meteomatics provides a lott of connectors to different program languages, for instance Python (that is free of charge)\cite{c1}.
    This makes it easier to implement the API in our application.
    \\\\
    Also, the response time of Meteomatics is faster than OpenWeatherMap.
    While it is not directly stated how fast Openweathermap response time is, it is known that it takes at most 1s, while Meteomatics is around 30ms.

    \subsubsection{Motivation}
    Meteomatics is a weather API that provides accurate and fast weather data.
    It has a free plan that offers more features than OpenWeatherMap, but it is limited to 500 calls per day.
    So in case we want to make more than 500 calls per day, we have to upgrade to a paid plan or choose for OpenWeatherMap.
    \\\\
    It also provides a lot of connectors to different program languages, which makes it easier to implement the API in our application,
    because we are split in teams and each team might have a different programming language this is a great advantage.
\end{document}