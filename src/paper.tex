\documentclass[conference]{IEEEtran}
\IEEEoverridecommandlockouts
% The preceding line is only needed to identify funding in the first footnote. If that is unneeded, please comment it out.
\usepackage{cite}
\usepackage{amsmath,amssymb,amsfonts}
\usepackage{algorithmic}
\usepackage{graphicx}
\usepackage{textcomp}
\usepackage{xcolor}
\usepackage{subfiles}
\def\BibTeX{{\rm B\kern-.05em{\sc i\kern-.025em b}\kern-.08em
    T\kern-.1667em\lower.7ex\hbox{E}\kern-.125emX}}
\begin{document}

\title{Methods for simulation of weather and climate
(mobility).\\}

\author{\IEEEauthorblockN{1\textsuperscript{st} Orfeo Tërkuçi}
\IEEEauthorblockA{\textit{Science} \\
\textit{University of Antwerp}\\
Boechout, Belgium \\
orfeo.terkuci@student.uantwerpen.be}
\and
\IEEEauthorblockN{2\textsuperscript{nd} Jason Liu}
\IEEEauthorblockA{\textit{Science} \\
\textit{University of Antwerp}\\
Lier, Belgium \\
jason.liu@student.uantwerpen.be}
}

\maketitle

\begin{abstract}
    In this paper we discuss about different methods for simulating weather and climate.
    These methods are STEP, OpenWeatherMap, Meteomatics, ClimaX, and GraphCast.
    We discuss about their advantages and disadvantages, and their motivation.
\end{abstract}

\section{Introduction}\label{sec:introduction}
\subfile{Sections/Introduction.tex}

\section{Topics}\label{sec:topics}

\subsection{STEP (short-term ensemble prediction system) computation algorithm}\label{subsec:step-(short-term-ensemble-prediction-system)-computation-algorithm}
\subfile{Sections/STEP.tex}

\subsection{Openweather API}\label{subsec:openweather-api}
\subfile{Sections/Openweather_API.tex}

\subsection{MeteoMatics}\label{subsec:meteomatics}
\subfile{Sections/MeteoMatics.tex}

\subsection{ClimaX}\label{subsec:climax}
\subfile{Sections/ClimaX.tex}

\subsection{GraphCast}\label{subsec:graphcast}
\subfile{Sections/GraphCast.tex}

\section{Conclusion}\label{sec:conclusion}
\subfile{Sections/Conclusion.tex}

\subfile{Sections/References.tex}

\end{document}
